\documentclass{beamer}
\usepackage[polish]{babel}
\usepackage[utf8]{inputenc}
\usepackage[T1]{fontenc}
\usepackage{ae}
\setbeamercovered{transparent}
\usetheme{Warsaw}

\begin{document}

\title{Problem: Minimum cardinality [2]-sumset cover}
\subtitle{Dotychczasowe wyniki i kwestie otwarte}
\author[Piotr Pytlik \& Jakub Sękowski]{Piotr Pytlik \and Jakub Sękowski}
\institute{Instytut Informatyki \\ Uniwersytet Wrocławski}

\begin{frame}
	\titlepage
\end{frame}

\begin{frame}
	\frametitle{Plan prezentacji}
	\tableofcontents
\end{frame}

\section{Minimum cardinality sumset covers}
	\subsection{Definicja}
		\begin{frame}
			def. h-iterated sumset,
		
			def. sumset (suma niesk. h-iterated sumset)
		
			def. problemu
		\end{frame}
		
	\subsection{Motywacja}
		\begin{frame}
			radioterapia
		\end{frame}	
			
	\subsection{NP-zupełność}
		\begin{frame}
			fakt o NP-zupełności
		\end{frame}
		
\section{Minimum cardinality [2]-sumset covers}
	\subsection{Definicja}
		\begin{frame} \frametitle{W czym rzecz?}
			Def.: [h]-sumset zbioru $ X $
			
			$ [h]X = \bigcup_{i=1}^{h} iX $
			
			Def.: [h]-sumset cover zbioru $ S \subset N^{+} $ to taki zbiór $ X \subset N^{+} $ że $ S \subset [h]X $			
			
			Problem: Minimum cardinality [h]-sumset cover
			
			Dla danego zbioru $ S \subset N^{+} $ znaleźć minimalny w sensie liczności [h]-sumset cover $ X $ zbioru $ S $. Wtedy $ |X| $ to [h]-sumset rank zbioru $ S $ (piszemy $ |X| = rk_{h}(S) $).
			
			Def.: Zbiór $ S \subset N^{+} $ jest [h]-simplifiable, jeśli $ rk_{h}(S) < |S| $
		\end{frame}
		
	\subsection{Przykłady}
		\begin{frame} \frametitle{Przykłady}
			Przykład 1:
			
			$ S = \lbrace 1,2,...,11 \rbrace $
			
			$ X = \lbrace 1,3,5,6 \rbrace $
		\end{frame}
			
		\begin{frame} \frametitle{Przykłady}
			Przykład 2:
			
			$ S = \lbrace 4,5,6 \rbrace $
			
			$ X = \lbrace 2,3 \rbrace $
		\end{frame}
		
		\begin{frame} \frametitle{Przykłady}		
			Przykład 3:
			
			$ S =  \lbrace 4,7,10 \rbrace $						
		\end{frame}
		
\section{Powiązane problemy}
	\subsection{Postage stamp}
		\begin{frame}
		\end{frame}
		
	\subsection{Covering the set of strings}
		\begin{frame}
		\end{frame}
		
\section{Trudność problemu}
	\subsection{Proste ograniczenia}
		\begin{frame}
			lemat 1 z "On finding..."
		\end{frame}
		
	\subsection{NP-trudność}
		\begin{frame} \frametitle{Nie miejmy złudzeń...}
			Fakt: Stwierdzenie, czy dany zbiór $ S \subset N^{+} $ jest [2]-simplifiable jest problemem NP-trudnym.
		\end{frame}
		
	\subsection{APX-trudność}
		\begin{frame} \frametitle{Aproksymacja też słabo...}
			Def.: Problem APX-trudny
		
			Fakt: Problem obliczania [h]-sumset rank zbioru $ S \subset N^{+} $ jest APX-trudny.
		\end{frame}
		
\section{Rozwiązanie}
	\subsection{Przypadek szczególny}
		\begin{frame} \frametitle{Przypadek $ S = \lbrace 1,2,...,n \rbrace $}
			Lemat: Niech $ S = \lbrace 1,2,...,n \rbrace $. Wtedy $ rk_{2}(S) = \Theta(\sqrt{n}) $
		\end{frame}
		
	\subsection{Znany algorytm}
		\begin{frame}
		\end{frame}
		
\section{Kwestie otwarte}
	\subsection{Silna NP-trudność}
		\begin{frame} \frametitle{Do zrobienia: silna NP-trudność}
			Def.: Problem silnie NP-trudny
			
			Pytanie: Czy problem obliczania [h]-sumset rank zbioru $ S \subset N^{+} $ jest silnie NP-trudny?
		\end{frame}
		
	\subsection{Ciekawa hipoteza}
		\begin{frame} \frametitle{Do zrobienia: hipoteza o $ S = \lbrace 1,2,...,n \rbrace $}
			Hipoteza: Dla każdej liczby naturalnej $ n $ istnieje minimum cardinality [2]-sumset cover $ X $ zbioru $ \lbrace 1,2,...,n \rbrace $ taki, że $ max(X) \leq \lceil \frac{n}{2} \rceil + 1 $
			
			Zweryfikowane dla $ n \leq 80 $, przy czym dla $ n \notin \lbrace 45,46,61,62 \rbrace $ dodatkowo $ max(X) \leq \lceil \frac{n}{2} \rceil $
		\end{frame}

\section{Bibliografia}
	\begin{frame} \frametitle{Nasze źródła}
		\begin{itemize}
			\item M.J. Collins, D. Kempe, J. Saia, and M. Young. Nonnegative integral
subset representations of integer sets. Information Processing Letters, 101, pp. 129–133, 2007.
			\item I. Fagnot, G. Fertin, S. Vialette. On finding small 2-generating sets. COCOON'2009, pp. 378–387.
			\item L. Bulteau, G. Fertin, R. Rizzi, S. Vialette. Some algorithmic results
for [2]-sumset covers. Information Processing Letters, 115, pp.1-5, 2015.
		\end{itemize}
	\end{frame}
	
\end{document} 
