\documentclass{beamer}
\usepackage[polish]{babel}
\usepackage[utf8]{inputenc}
\usepackage[T1]{fontenc}
\usepackage{ae}
\setbeamercovered{transparent}
\usetheme{Warsaw}

\begin{document}

\title{Problem: Minimum cardinality [2]-sumset cover}
\subtitle{Dotychczasowe wyniki i kwestie otwarte}
\author[Piotr Pytlik \& Jakub Sękowski]{Piotr Pytlik \and Jakub Sękowski}
\institute{Instytut Informatyki \\ Uniwersytet Wrocławski}

\begin{frame}
	\titlepage
\end{frame}

\begin{frame}
	\frametitle{Plan prezentacji}
	\tableofcontents
\end{frame}

\section{Minimum cardinality sumset covers}
	\subsection{Definicja}
	\subsection{Motywacja}
	\subsection{NP-zupełność}
\section{Minimum cardinality [2]-sumset covers}
	\subsection{Definicja}
	\subsection{Przykłady}
\section{Powiązane problemy}
	\subsection{Postage stamp}
	\subsection{Covering the set of strings}
\section{Trudność problemu}
	\subsection{Proste ograniczenia}
	\subsection{NP-trudność}
	\subsection{APX-trudność}
\section{Rozwiązanie}
	\subsection{Przypadek szczególny}
	\subsection{Znany algorytm}
\section{Kwestie otwarte}
	\subsection{Silna NP-trudność}
	\subsection{Ciekawa hipoteza}

\end{document} 
