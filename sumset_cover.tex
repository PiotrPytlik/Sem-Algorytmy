\documentclass{beamer}
\usepackage[polish]{babel}
\usepackage[utf8]{inputenc}
\usepackage[T1]{fontenc}
\usepackage{ae}
\usepackage{amsfonts}
\usepackage{algorithm}
\usepackage[noend]{algpseudocode}
\usetheme{Warsaw}

\setbeamertemplate{headline}{}
\setbeamercovered{invisible}

\algrenewcommand\algorithmicrequire{\textbf{Dane:}}
\algrenewcommand\algorithmicensure{\textbf{Wynik:}}

\newcommand{\N}{\mathbb{N}}

\begin{document}

\title{Minimalne [2]-sumsetowe pokrycie zbioru}
\subtitle{Dotychczasowe wyniki i kwestie otwarte}
\author[Piotr Pytlik \& Jakub Sękowski]{Piotr Pytlik \and Jakub Sękowski}
\institute{Instytut Informatyki \\ Uniwersytet Wrocławski}
\date{22 grudnia 2014}

\begin{frame}
	\titlepage
\end{frame}

\begin{frame}{Plan prezentacji}
	\setcounter{tocdepth}{1}
	\tableofcontents
\end{frame}

\section{Sumsetowe pokrycia o minimalnej mocy}
	\subsection{Definicja}
		\begin{frame}{Ogólniejszy problem}
			\begin{alertblock}{Pokrycie sumsetowe o minimalnej mocy (\rmfamily \textsc{Minimum Generating Set})}
                Dla danego zbioru $ S \subset \N^{+} $ znaleźć taki zbiór X, że każdy element zbioru S jest sumą pewnego podzbioru zbioru X, i ponadto X miał najmniejszą moc spośród zbiorów o tej własności.
			\end{alertblock}
		\end{frame}
		
	\subsection{Motywacja}
		\begin{frame}{Do czego to się przydaje}
            \begin{block}{Radioterapia}
    				Elementy z $ S $ oznaczają dawki promieniowania potrzebne w różnych miejscach na ciele, a elementy z $ X $ oznaczają dawki, które mogą być podawane w kilku miejscach na ciele jednocześnie.
            \end{block}
		\end{frame}	
			
	\subsection{NP-zupełność}
		\begin{frame}{Wcale nie takie proste}
            \begin{block}{NP-zupełność}
                Udowodniono w pracy \emph{Nonnegative Integral Subset Representations of Integer Sets} przez M.J. Collins, D. Kempe, J. Saia, M. Young.
                Pokazali oni redukcję z problemu \emph{Positive NAE-3SAT(5)}, który jest NP-zupełny, do problemu:
            \end{block}
            \begin{alertblock}{Generating Set}
                Mając podany zbiór $ S \subset \N^{+} $ oraz $ k > 0 $ sprawdzić czy istnieje zbiór $ X \subset \N^{+} $ o mocy $ k $ taki, że każdy element zbioru S jest sumą pewnego podzbioru zbioru X.
            \end{alertblock}
		\end{frame}
		
\section{[2]-sumsetowe pokrycia o minimalnej mocy}
	\subsection{Definicja problemu}
		\begin{frame}{Definicja problemu}
			\begin{block}{Definicje}
				\begin{itemize}
					\item \emph{h-iterowany sumset} zbioru $ X $ to zbiór $ hX = \lbrace x_1 + x_2 + ... + x_h : x_1,x_2,...,x_h \in X \rbrace $
					\item \emph{[h]-sumset} zbioru X to zbiór $ [h]X = \bigcup_{i=1}^{h} iX $
					\item zbiór $ X \subset \N^{+} $ jest \emph{[h]-sumsetowym pokryciem} zbioru $ S \subset \N^{+} $, jeśli $ S \subset [h]X $
				\end{itemize}
			\end{block}
			
			\pause
			\begin{alertblock}{[h]-sumsetowe pokrycie o minimalnej mocy (\rmfamily \textsc{Minimum cardinality [h]-sumset cover problem})}
				Dla danego zbioru $ S \subset \N^{+} $ znaleźć [h]-sumsetowe pokrycie X zbioru S o jak najmniejszej mocy.
				
				Wtedy $ |X| $ to \emph{[h]-sumsetowy rząd} zbioru S (piszemy $ |X| = rk_{h}(S) $).
			\end{alertblock}
			
			\pause
			W tej prezentacji zajmiemy się przypadkiem $ h=2 $.
		\end{frame}
		
	\subsection{Przykłady}
		\begin{frame}{Przykłady}
			\begin{enumerate}
				\item $ S = \lbrace 1,2,...,11 \rbrace $, $ X = \lbrace 1,3,5,6 \rbrace $
				\pause \item $ S = \lbrace 4,5,6 \rbrace $, $ X = \lbrace 2,3 \rbrace $
				\pause \item $ S =  \lbrace 4,7,10 \rbrace $
			\end{enumerate}						
		\end{frame}
		
\section{Powiązane problemy}
	\subsection{Znaczki pocztowe}
		\begin{frame}{Znaczki pocztowe}
			\begin{block}{Znaczki pocztowe (\rmfamily \textsc{Postage Stamp Problem})}
                Mając podane $ h, k \in \N^{+} $, znaleźć największe takie $ n $, dla którego istnieje [h]-sumsetowe pokrycie $ X $ dla zbioru $ \lbrace 1, 2, \dots , n \rbrace $, o mocy $ k $. Takie $ n $ oznaczamy jako $ N(h,k) $.
                \begin{itemize}
				    \pause \item $ N(1,5) = 5 $, $ X = \lbrace 1, 2, 3, 4, 5 \rbrace $
				    \pause \item $ N(13,1) = 13 $, $ X = \lbrace 1 \rbrace $
				    \pause \item $ N(3,2) = 7 $, $ X = \lbrace 1, 3 \rbrace $
				    \pause \item $ N(2,3) = 8 $, $ X = \lbrace 1, 3, 4 \rbrace $
                    \pause
                \end{itemize}
            \end{block}
            \begin{block}{Ciekawe przypadki}
                Łatwo zauważyć, że $ N(1,k) = k $, oraz  $ N(h,1) = h $.
                Udowodniono również, że $ N(h,2) = \lfloor \frac{h^2 + 6h + 1}{4} \rfloor $.
                Zaskakujące jest jednak to, że nadal nie istnieją zwięzłe wzory na $ n $ dla żadnych innych par $ (h,k) $ gdzie jeden z wyrazów jest ustalony, w tym też dla $ N(2,k) $.
			\end{block}
		\end{frame}
		
	\subsection{Pokrycie zbioru słów}
		\begin{frame}{Pokrycie zbioru słów}
%			Tutaj tez trzeba troche wiecej szczegolowosci. Problem jest NP-trudny dla alfabetu binarnego, ale dla unarnego jego status nie jest znany (!). Jest to jeden z problemow otwartych, o ktorych mowimy na koniec prezentacji. Poprawilem to po swojemu, ale w razie czego, mozesz jeszcze to modyfikowac.

			\begin{alertblock}{l-pokrycie zbioru słów}
				Dla danego zbioru słów $ S $, znaleźć możliwie najmniej liczny zbiór słów $ C $ taki, że każde słowo ze zbioru $ S $ jest konkatenacją co najwyżej $ l $ słów ze zbioru $ C $.
			\end{alertblock}
			
			\pause
			\begin{exampleblock}{Fakt}
					Jeśli rozważamy słowa nad pewnym alfabetem binarnym, oraz $ l \geq 2 $, to problem jest NP-trudny.
			\end{exampleblock}
			
			\pause
			Dla nas interesujący jest przypadek alfabetu unarnego. Wtedy problem sprowadza się do [l]-sumsetowego pokrycia zbioru.
			
			Kilka przykładów:
            	\begin{enumerate}
				\pause \item $ S = \lbrace a,aa,aaa,\dots,aaaaaaaaaaa \rbrace $, $ C = \lbrace a,aaa,aaaaa,aaaaaa \rbrace $
				\pause \item $ S = \lbrace aaaa,aaaaa,aaaaaa \rbrace $, $ C = \lbrace aa,aaa \rbrace $
				\pause \item $ S = \lbrace aaaa,aaaaaaa,aaaaaaaaaa \rbrace $, $ C = S $
            \end{enumerate}
		\end{frame}
		
\section{Trudność problemu}
	\subsection{Proste ograniczenia}
		\begin{frame}{Proste ograniczenia na [2]-sumsetowy rząd}
            \begin{block}{Lemat}
			    Dla dowolnego $ S \subset \N^{+} $ o mocy $ n $:
                $ \lceil \frac{1}{2} ( \sqrt{8n + 9} - 3 ) \rceil \leq rk_2(S) \leq n $
            \end{block}

            \pause
            \vspace{\baselineskip}
            Warto zanotować sobie, że $ rk_2(S) = \Omega \left( \sqrt{n} \right) $, gdyż z tego będziemy jeszcze korzystali.
		\end{frame}
		
	\subsection{NP-trudność}
		\begin{frame}{Nie miejmy złudzeń...}
			\begin{block}{Definicja}
				Zbiór $ S \subset \N^{+} $ jest \emph{[h]-upraszczalny}, jeśli $ rk_{h}(S) < |S| $
			\end{block}
			
			\begin{exampleblock}{Fakt}
				Stwierdzenie, czy dany zbiór $ S \subset \N^{+} $ jest [2]-upraszczalny jest problemem NP-trudnym.
			\end{exampleblock}
		\end{frame}
		
	\subsection{APX-trudność}
		\begin{frame}{Aproksymacja też słabo...}
			\begin{exampleblock}{Fakt}
				Problem obliczania [h]-sumsetowego rzędu zbioru $ S \subset \N^{+} $ jest APX-trudny.
			\end{exampleblock}			
			
			\pause
			\begin{block}{z Wikipedii:}
				Problem jest \emph{APX-trudny}, jeśli dla każdego problemu w klasie APX istnieje do niego PTAS-redukcja.
			\end{block}						
			
			\pause
			\begin{exampleblock}{Fakt (też z Wikipedii)}
				Jeśli dla pewnego problemu APX-trudnego istnieje wielomianowy algorytm rozwiązujący go z dowolnie małym błędem multiplikatywnym, to $ P = NP $.
			\end{exampleblock}
			
		\end{frame}
		
\section{Rozwiązanie}
	\subsection{Przypadek szczególny}
		\begin{frame}{Przypadek $ S = \lbrace 1,2,...,n \rbrace $}
			\begin{exampleblock}{Lemat}			
				Niech $ S = \lbrace 1,2,...,n \rbrace $. Wtedy $ rk_{2}(S) \leq \left\lceil \sqrt{n} \right\rceil - 2 $
			\end{exampleblock}
						
			
			\pause			
			\vspace{\baselineskip}
			\begin{exampleblock}{Wniosek}			
				$ rk_{2}\left( \lbrace 1,2,...,n \rbrace \right) = \Theta(\sqrt{n}) $
			\end{exampleblock}
		\end{frame}
		
	\subsection{Algorytm FPT dla przypadku ogólnego}
		\begin{frame}{Algorytmy FPT}
			\begin{block}{Definicja}			
				Niech n będzie rozmiarem danych pewnego problemu, a k jednym z parametrów. Algorytm rozwiązujący ten problem w czasie $ O(f(k)poly(n)) $ dla pewnej funkcji f jest \emph{FPT (Fixed parameter tractable)}.
			\end{block}
			
			\pause
			\begin{exampleblock}{Wniosek}			
				Przy ustalonej wartości parametru k, algorytm FPT działa w czasie wielomianowym.
			\end{exampleblock}
		\end{frame}

		\begin{frame}{Wstęp}
            \begin{block}{Wcześniejszy algorytm}
                W pracy \emph{On finding small 2-generating sets} przedstawiono lemat o reprezentacji, z którego wynikał algorytm \emph{FPT} działający w czasie
                $ O\left( 5^{\frac{k^{2}(k+3)}{2}}k^{2}\log k \right) $ sprawdzający czy istnieje [2]-sumsetowe pokrycie zbioru $ S \subset \N^{+} $ o mocy $ \leq k $.
            \end{block}
            \begin{block}{Nowy algorytm}
                Przedstawimy lepszy algorytm \emph{FPT} działający w czasie
                $ O\left( 2^{k(3\log k-1.4)}poly(k) \right)$
            \end{block}
        \end{frame}
        \begin{frame}{Opis}
            \begin{block}{Opis danych}
                Mając multigraf $ G = (V,E) $ oznaczamy
                \begin{itemize}
                    \item $ N_{G}(u) $ - zbiór sąsiadów $ u \in V $
                    \item $ \forall V' \subset V, N_{G}(V') $ - zbiór sąsiadów wszystkich wierzchołków $ V' $
                \end{itemize}
                W obu przypadkach dopuszczamy zbiory z powtórzeniami.
                Niech $ S = \lbrace s_i : 1 \leq i \leq n \rbrace \subset \N^{+} $, a niech $ X = \lbrace x_i : 1 \leq i \leq k \rbrace \subset \N^{+} $ będzie jakimś [2]-sumsetowym pokryciem $ S $.
            \end{block}
        \end{frame}
        \begin{frame}{Definicje}
            \begin{block}{$ X $-realizacja $ S $}
                Mówimy, że graf dwudzielny $ B = (U,V,E) $ jest $ X $-realizacją $ S $ z dwoma bijekcjami $ \sigma : U \rightarrow S, \chi \rightarrow X $ jeżeli $ \forall u \in U, d_{B}(u) \in \lbrace 1,2 \rbrace $ oraz:
                \begin{itemize}
                    \item $ d_{B}(u) = 1 $ i $ \lbrace u, v \rbrace \in E $, to $ \sigma(u) = \chi(v) $
                    \item $ d_{B}(u) = 2 $ i $ \lbrace u, v \rbrace, \lbrace u, v' \rbrace \in E $, to $ \sigma(u) = \chi(v) + \chi(v') $ (być może $ v = v' $)
                \end{itemize}
            \end{block}
            \begin{block}{Minimalna $ X $-realizacja $ S $}
                $ B = (U,V,E) $, będące $ X $-realizacją $ S \subset \N^{+} $ jest minimalna, jeśli $ X $ jest [2]-sumsetowym pokryciem $ S $ o minimalnej mocy, tzn. $ S $ ma [2]-sumsetowy rząd $ |X| $.
            \end{block}
		\end{frame}
		
		\begin{frame}{Pomysł na algorytm}
			\begin{exampleblock}{Lemat 1}
                Niech $ B = (U,V,E) $ będzie minimalną $ X $-realizacją pewnego $ S \subset \N^{+} $.
                Każda spójna składowa posiada $ u \in U $ stopnia 1 i/lub cykl prosty o długości $ 4l + 2 $ dla pewnego $ l \geq 0 ) $
            \end{exampleblock}
		\end{frame}
		
		\begin{frame}{Pomysł na algorytm}
            \begin{block}{Graf $ k $-zredukowany}
                $ B = (U,V,E) $ jest $ k $-zredukowany jeśli:
                \begin{enumerate}
                    \pause \item $ |U| = |V| = k $
                    \pause \item $ \forall u \in U, d_{B}(u) \in \lbrace 1, 2 \rbrace $
                    \pause\item $ B $ zawiera pełne skojarzenie
                    oraz
                    \pause \item każda spójna składowa ma pewien $ u \in U, d_{B}(u) = 1 $ i/lub cykl prosty o długości $ 4l + 2 $ dla pewnego $ l \geq 0 $.
                \end{enumerate}
            \end{block}
		\end{frame}

		\begin{frame}{Pomysł na algorytm}
			\begin{exampleblock}{Lemat 2}
                Niech $ B = (U,V,E) $ będzie $ k $-zredukowanym grafem dwudzielnym oraz $ \sigma : U \rightarrow \N^{+} $ będzie iniekcją.
                Wtedy istnieje najwyżej jedno mapowanie $ \chi : V \rightarrow \N^{+} $ t. że dla dowolnego $ u \in U $ mamy $ \sigma(u) = \sum_{v \in N(u)}\chi(v) $. Możemy też wyliczyć to mapowanie w czasie wielomianowym, o ile ono istnieje.
            \end{exampleblock}
		\end{frame}
		
		\begin{frame}{Pomysł na algorytm}
            \begin{block}{Dowód lematu 2}
                Najpierw wyliczamy $ \chi(v^{+}) $ dla dokładnie jednego $ v^{+} \in V $ w każdej spójnej składowej w $ B $.
                \begin{itemize}
                    \item Niech $ B_i = (U_i,V_i,E_i) $ będzie dowolną spójną składową $ B $. Jeśli istnieje $ u \in U_i $ stopnia 1, niech $ v^{+} $ będzie (jedynym) sąsiadem $ u $ oraz ustalmy $ \chi(v^{+}) = \sigma(u) $.
                    \pause \item W.p.p. zgodnie z lematem 1, $ B_i $ ma cykl prosty długości $ 4l + 2 $ dla pewnego $ l \geq 0 $.
                        Niech $ \left( u_1,v^{+},u_2,v_2, \dots, u_{2l+1}, v_{2l+1} \right) $ będzie dowolnym takim cyklem, i ustalmy $ \chi(v^{+}) = \sum_{i=1}^{2l+1}\frac{1}{2}(-1)^{i-1}\sigma(u_i) $.
                \end{itemize}
            \end{block}
        \end{frame}
        \begin{frame}{Pomysł na algorytm}
            \begin{block}{Dowód lematu 2 - c.d.}
                Teraz dla każdej spójnej składowej jest dokładnie jeden $ v^{+} $ z ustalonym $ \chi(v^{+}) $.
                \pause
                \begin{itemize}
                    \item $ \forall v \in V $ że $ \chi(v) $ nie zostało ustalone, rozpatrzmy dowolną drogę długości $ 2p $ od $ v $ do $ v^{+} \in V $, że $ \chi(v^{+}) $ było ustalone w kroku 1, $ \left( v,u_1,v_1, \dots, v_{p-1},u_p,v^{+} \right) $. Ustalmy $ \chi(v) = (-1)^p\chi(v^{+}) + \sum_{i=1}^{p}(-1)^{i-1}\sigma(u_i) $.
                \end{itemize}
                Teraz $ \forall u \in U $ wystarczy sprawdzić, że $ \sigma(u) = \sum_{v \in N(u)}\chi(v) $.
            \end{block}
            \begin{block}{Wnioski}
                Widać, że ten program pracuje w czasie wielomianowym.
                \newline
                Jeżeli $ \chi $ ma dodatnie, całkowitoliczbowe wartości, to $ \chi : V \rightarrow \N^{+} $ jest poszukiwanym mapowaniem.
                \newline
                Ponadto z konstrukcji wynika, że jest to jedyne mapowanie spełniające $ \forall u \in U, \sum_{v \in N(u)}\chi(v) = \sigma(u) $.
            \end{block}
		\end{frame}

		\begin{frame}{Pomysł na algorytm}
			\begin{exampleblock}{Lemat 3}			
				Niech X będzie [2]-sumsetowym pokryciem zbioru $ S \subset \N^{+} $ o minimalnej mocy ($ |X| = k $) i niech $ B = (U,V,E) $ będzie minimalną X-realizacją zbioru S. Wtedy istnieje podzbiór $ U' \subset U $ o mocy k taki, że podgraf indukowany $ B[U',V] $ jest k-zredukowany.
			\end{exampleblock}
		\end{frame}
				
		\begin{frame}{Pomysł na algorytm}
			\begin{algorithm}[H]
			\begin{algorithmic}[1]
				\Require $ S \subset \N^{+} $, $ k \in \N^{+} $
				\Ensure 	[2]-sumsetowe pokrycie zbioru S o mocy k, jeśli takie istnieje lub \emph{NIL} wpp
				\Statex
				\ForAll{dwudzielne k-zredukowane grafy $ B = (U,V,E) $}
					\ForAll{podzbiory $ S' \subset S $ o mocy k}
						\ForAll{bijekcje $ \sigma: U \rightarrow S' $}
							\If{istnieje odpowiednia bijekcja $ \chi: V \rightarrow \N^{+} $}
								\State $ X \gets \lbrace \chi(v) : v \in V \rbrace $ \Comment{patrz: Lemat 2}
								\If{$ S \subset [2]X $}
									\State \Return X
								\EndIf
							\EndIf
						\EndFor
					\EndFor					
				\EndFor
				\State \Return \emph{NIL}
			\end{algorithmic}
			\end{algorithm}
		\end{frame}
		
		\begin{frame}{Poprawność algorytmu}
			Poprawność algorytmu wynika z lematu 2 oraz następującego faktu.		
		
			\begin{exampleblock}{Wniosek z lematu 3}			
				Niech $ S \subset \N^{+} $ będzie zbiorem, dla którego $ rk_{2}(S) = k $. Wtedy istnieje podzbiór $ S' \subset S $ o mocy k taki, że jego X-realizacja jest k-zredukowana.
			\end{exampleblock}
		\end{frame}
		
		\begin{frame}{Złożoność czasowa}
			\begin{itemize}
				\item wszystkich k-zredukowanych grafów B jest $ O\left( k^{3/2 + k}e^{-k} \right) $
				\pause \item wszystkich k-elementowych podzbiorów S' zbioru S jest $  \binom{|S|}{k} $
				\pause \item wszystkich bijekcji $ U \rightarrow S' $ jest $ k! $
				\pause \item znalezienie bijekcji $ \chi: V \rightarrow \N^{+} $ jest $ O\left( poly(k) \right) $ (z Lematu 2)
				\pause \item sprawdzenie warunku $ S \subset [2]X $ również jest $ O\left( poly(k) \right) $
			\end{itemize}
			
			\pause
			Korzystając z prostego ograniczenia $ |S| = O\left( k^2 \right)$, mamy:
			
			\begin{exampleblock}{Fakt}			
				Algorytm działa w $ O\left( k^{3k}e^{-k}poly(k) \right) = O\left( 2^{k(3\log k-1.4)}poly(k) \right)$
			\end{exampleblock}		
		\end{frame}
\section{Kwestie otwarte}
	\subsection{Silna NP-trudność}
		\begin{frame}{Do zrobienia: silna NP-trudność}
			\begin{block}{Definicja}
				\emph{Problem silnie NP-trudny} to problem, który pozostaje NP-trudny nawet, gdy wszystkie jego dane liczbowe są ograniczone przez wielomian rozmiaru danych.
				
				\pause \textbf{Równoważnie:} Istnieje do niego redukcja wielomianowa z jakiegoś problemu silnie NP-zupełnego.
			\end{block}				
			
			\pause
			\begin{alertblock}{Problem otwarty}
				Czy problem obliczania [2]-sumsetowego rzędu zbioru $ S \subset \N^{+} $ jest silnie NP-trudny?
			\end{alertblock}
		\end{frame}
		
	\subsection{Ciekawa hipoteza}
		\begin{frame}{Do zrobienia: hipoteza o $ S = \lbrace 1,2,...,n \rbrace $}
			\begin{alertblock}{Hipoteza}
				Dla każdej liczby naturalnej n istnieje [2]-sumsetowe pokrycie X o minimalnej mocy zbioru $ \lbrace 1,2,...,n \rbrace $ takie, że $ max(X) \leq \lceil \frac{n}{2} \rceil + 1 $
			\end{alertblock}
			
			\pause
			\vspace{\baselineskip}
			Zweryfikowane dla $ n \leq 80 $, przy czym dla $ n \notin \lbrace 45,46,61,62 \rbrace $ dodatkowo $ max(X) \leq \lceil \frac{n}{2} \rceil $
		\end{frame}

\section{Bibliografia}
	\begin{frame}{Nasze źródła}
		\begin{itemize}
			\item M.J. Collins, D. Kempe, J. Saia, and M. Young. \textit{Nonnegative integral
subset representations of integer sets}. Information Processing Letters, 101, pp. 129–133, 2007.
			\item I. Fagnot, G. Fertin, S. Vialette. \textit{On finding small 2-generating sets}. COCOON'2009, pp. 378–387.
			\item L. Bulteau, G. Fertin, R. Rizzi, S. Vialette. \textit{Some algorithmic results
for [2]-sumset covers}. Information Processing Letters, 115, pp.1-5, 2015.
			\item Angielska Wikipedia
		\end{itemize}
	\end{frame}
	
\end{document}
